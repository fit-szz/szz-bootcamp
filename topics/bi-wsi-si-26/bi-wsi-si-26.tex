\documentclass{szzclass}
\usepackage{hyperref}
\usepackage{longtable}
\usepackage{booktabs}

\subject{SI2}
\code{BI-WSI-SI-26}
\topic{Údržba: život softwarového díla, role a význam údržby, vazba na SDLC a jednotlivé činnosti softwarového inženýrství, servisní smlouva, role měření (pracnosti, nákladů a dalších metrik) při údržbě.}
\providecommand{\tightlist}{%
  \setlength{\itemsep}{0pt}\setlength{\parskip}{0pt}}

\begin{document}
\tableofcontents
\newpage

\section{Život SW díla}
\begin{itemize}
    \item před nasazením SW kroky jako analýza, design a implementace
    \item po nasazení SW následuje údržba a další rozvoj
\end{itemize}
Čas po kterou je produkt udržován je několikanásobně delší než vývoj. Jednotlivé kroky za sebou:
\newline
Inception $\rightarrow$ elaboration $\rightarrow$ construction $\rightarrow$ transition $\rightarrow$ mantenance
\newline
Typy údržby:
\begin{itemize}
    \item corrective (opravné) - oprava nalezených chyb a problémů
    \item adaptive (adaptivné) - udržení v měnícím se prostřdedí
    \item perfective (zdokonalovací) - zlepšování výkonnosti nebo udržitelnosti
    \item preventive (preventivní) - detekce a oprava chyb než se stanou skutečné
\end{itemize}

\section{SDLC}
Při změně SW zpravidla potřebujeme procést celý vývojový cyklus znovu (analýza, design, implementace, testování, dodávka) - miniwaterfall.
\newline
Na rozdíl od počátečního vývoje je cyklus značně redukován, protože se zaměřuje pouze na nově dodávanou funkcionalitu.
\newline
Pokud jsme zároveň tvůrci produktu, je miniwaterfall velmi efektivní (system známe, tým vyvíjející původní SW se zároveň stará o jeho údržbu).
\newline
S rostoucím rozdílem mezi týmem vyvíjející a spravující systém je efektivita miniwaterfallu přímo úměrná kvalitě dokumentace projektu.
\newline
Snadno se měří náklady na údržbu - přesné odhady pro uzavření servisní smlouvy

\section{Správa}
\subsection{Problémy}
Více různých prostředí:
\begin{itemize}
    \item vývojové - oprava chyb, vývoj nových funkčností
    \item testovací - ověření oprav, kontrola neovlivnění ostatních částí
    \item produkční
    \item školící
    \item \dots
\end{itemize}
Ideální stav, když se jedná o totožné prostředí, ale dochází k problémům s:
\begin{itemize}
    \item licencemi
    \item HW požadavky
    \item testovacími daty (obsah osobních dat/příliš velká)
\end{itemize}
Velikost testovacích dat, dělají se řezy dat:
\begin{itemize}
    \item zmenšní požadavku na kapacitu disku
    \item zajistit reprezentativní vzorek dat (problém podle čeho data ožíznout)
    \item zajistit konzistenci dat (nutné neporušit referenční integritu)
    \item problém pro výkonnostní testy
\end{itemize}
Anonymizace:
\begin{itemize}
    \item odstranic citlivé údaje - co vše je citlivé?
    \item zachovat délky řezů, formáty dat, závislost mezi tebulkami, odlišný obsah v závislosti na jiném údaji, vnitřní logika (např. rodné číslo)
\end{itemize}
Údržba cizího systému přináší mnoho problémů, je nutná dokumentace, vytvořit si znalostní bázi o řešení problémů a je nebezpečí zníčení původní architektury (architekturu navrhl někdo jiný).
\newline
Dodávky se rozdělují podle verzí s novými funkčnostmi a těmi, které opravují akutní opravy. Je potřeba používat CM (konfigurační řízení) pro verzování současné práce na
opravách a vývoji nových funkcí.
\subsection{Konfigurační řízení}
Evidence nových požadavků, změn a chyb. Plánování nasazování nových verzí.
Definuje process změnového řízení a eviduje všechny požadavky zákazníka.
\subsection{Testování}
Především regresní testy, protože testování celého systému je náročné.
\subsection{Odhady}
Přesné odhady jsou klíčové, aby udržba byla profitabilní. Musí být konzistetní a odchylky musíme být schopni zdůvodnit.
\subsection{Metriky}
Jsou klíčové pro odhady a jsou základem ceny servisní smlouvy. Sleduje se rozsah změn v MD a poměr změn vůči původní velikosti projektu.

\section{Podpora}
Jedná se o službu pro uživatele systému, umožňuje řešit problémy, které se během používání objeví.
Typy podpory:
\begin{itemize}
    \item po telefonu
    \item po emailu
    \item po internetu
\end{itemize}
Podpora se dělí do několika úrovní:
\begin{itemize}
    \item 1. úroveň
    \begin{itemize}
        \item komunikuje se zákazníkem a snaží se identifikovat příčiny problému
        \item řešení hledá v databázi možných řešení, která se během provozu podpory vytváří
        \item vyřeší většinu problémů
    \end{itemize}
    \item 2. úroveň
    \begin{itemize}
        \item technická úroveň
        \item hlubší znalost daného produktu a technologie
        \item mohou být osoby z vývojového týmu
    \end{itemize}
    \item řeší nejvíce obtížné případy
    \item experti v dané oblasti
\end{itemize}
\subsection{Shrnutí}
Typicy se jedná o dlouhodobé smlouvy. Podpora zajišťuje firmě stabilní množství práce/peněz. Velkou výhodou je dobrá znalsot systému,
protože se dobře dělají odhady pracnosti nových funkcí a náročnost provádění údržby.
\section{SLA - servisní smlouva}
V rámci provozu systému a jeho podpory jsou garantovány určitě parametry.
\begin{itemize}
    \item dostupnost - jak budeme k dispozici
    \item stabilita
    \item response time - za jak dlouhou dobu musíme odpověď na dotaz
    \item fix time - za jak dlouho musí být chyba opravena
\end{itemize}
Při porušení SLA hrozí sankce od zákazníka. Základem je nedeklarovat parametry, které nemůžeme ovlivnit.
\newline
Podpora se dá rozdělit podle doby dostupnosti (24/7, 8*5,\dots) nebo podle místa, kde se podpora koná (on-site: u zákazníka / on-call: po telefonu)

\end{document}