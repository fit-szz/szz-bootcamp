\documentclass{szzclass}
\usepackage{hyperref}
\usepackage{longtable}
\usepackage{booktabs}

\subject{SI2}
\code{BI-WSI-SI-24}
\topic{Údržba: život softwarového díla, role a význam údržby, vazba na SDLC a jednotlivé činnosti softwarového inženýrství, servisní smlouva, role měření (pracnosti, nákladů a dalších metrik) při údržbě.}
\providecommand{\tightlist}{%
  \setlength{\itemsep}{0pt}\setlength{\parskip}{0pt}}

\begin{document}
\tableofcontents
\newpage

\section{Život SW díla}
\begin{itemize}
    \item před nasazením SW kroky jako analýza, design a implementace
    \item po nasazení SW následuje údržba a další rozvoj
\end{itemize}
Čas po kterou je produkt udržován je několikanásobně delší než vývoj. Jednotlivé kroky za sebou:
\newline
Inception $\rightarrow$ elaboration $\rightarrow$ construction $\rightarrow$ transition $\rightarrow$ mantenance
\newline
Typy údržby:
\begin{itemize}
    \item corrective - oprava nalezených chyb a problémů
    \item adaptive - udržení v měnícím se prostřdedí
    \item perfective - zlepšování výkonnosti nebo udržitelnosti
    \item preventive - detekce a oprava chyb než se stanou skutečné
\end{itemize}

\section{SDLC}
Při změně SW zpravidla potřebujeme procést celý vývojový cyklus znovu (analýza, design, implementace, testování, dodávka) - miniwaterfall.
\newline
Na rozdíl od počátečního vývoje je cyklus značně redukován, protože se zaměřuje pouze na nově dodávanou funkcionalitu.
\newline
Pokud jsme zároveň tvůrci produktu, ne miniwaterfall velmi efektivní (system známe, tým vyvíjející původní SW se zároveň stará o jeho údržbu).
\newline
S rostoucím rozdílem mezi týmem vyvíjející a spravující systém je efektivita miniwaterfallu přímo úměrná kvalitě dokumentace projektu.

\subsection{Konfigurační řízení}
Definuje process změnového řízení a eviduje všechny požadavky zákazníka.
\subsection{Testování}
Především regresní testy, protože testování celého systému je náročné.
\subsection{Odhady}
Přesné odhady jsou klíčové, aby udržba byla profitabilní. Musí být konzistetní a odchylky musíme být schopni zdůvodnit.
\subsection{Metriky}
Jsou klíčové pro odhady a jsou základem ceny servisní smlouvy. Sleduje se rozsah změn v MD a poměr změn vůči původní velikosti projektu.
\section{SLA - servisní smlouva}
V rámci provozu systému a jeho podpory jsou garantovány určitě parametry.
\begin{itemize}
    \item dostupnost - jak budeme k dispozici
    \item stabilita
    \item response time - za jak dlouhou dobu musíme odpověď na dotaz
    \item fix time - za jak dlouho musí být chyba opravena
\end{itemize}
Při porušení SLA hrozí sankce od zákazníka. Základem je nedeklarovat parametry, které nemůžeme ovlivnit.
\newline
Podpora se dá rozdělit podle doby dostupnosti (24/7, 8*5,\dots) nebo podle místa, kde se podpora koná (on-site: u zákazníka / on-call: po telefonu)

\end{document}