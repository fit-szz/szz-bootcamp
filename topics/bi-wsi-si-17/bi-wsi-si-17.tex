\documentclass{szzclass}

\usepackage[czech]{babel}

\topic{Adresace IPv4, rozdělení adresního prostoru. Rozdíly mezi IPv4 a IPv6.}
\author{Tomáš Starý}
\code{BI-WSI-SI-17}
\subject{PSI}

\begin{document}

\tableofcontents

\newpage

\section{Adresace}

Abychom mohli komunikovat mezi jednotlivými uzli v síti je nutné nějdříve každému uzlu přidělit nějakou adresu, díky které jej bude možné identifikovat.
Každá síť má definovanou vlastní množinu adresu, která je disjunktní s ostatními sitěmi (lze tedy narazit na 2 sítě s totožnou adresací). V každé síti
je poté potřeba rozdělit její adresní prostor případně celou síť navenek zapouzdřit jednou adresou (NAT) a tím vytvořit podsítě (víceúrovňové uspořádání).

\section{IPv4}

Hlavní význam IPv4 v počítáčových sítích je adresace jednotlivých uzlů v síti. Díky této adresaci je poté možné
předat data správnému adresátovi. IPv4 je 32 bitová adresa, která slouží k indentifikaci síťových zařízení na 3. úrovni modelu
ISO-OSI (síťová vrstva). IPv4 je rozdělena do 4 oktetů (bytů), adresy se dají zapsat také dekadicky, kde každý oktet je od dalšího oddělen tečkou.

\subsection{Adresace pomocí IPv4}

Původní rozdělení IPv4 bylo pomocí tříd
\begin{itemize}
    \item A, B, C - normální adresy (unicast)
    \item D - multicast
    \item E - rezervováno
\end{itemize}

\begin{tabular}{c|c|c|c|c|c}
    Třída & Úvodní bity & síťových adres (bitů) & počet sítí & počet adres v síti & první adresa \\
    \hline
    A & 0 & 8 & 128 & $2^24$ & 0.0.0.0 \\
    B & 10 & 16 & 16384 & $2^16$ & 128.0.0.0 \\
    C & 110 & 24 & 2097152 & $2^8$ & 192.0.0.0 \\
    D & 1110 & & & & 224.0.0.0 \\
    E & 1111 & & & &  240.0.0.0 \\
\end{tabular}

S tímto rozdělením adres přišel ovšem rychle problém jejich neefektivnosti, v některých případech bylo nutné jít o třídu víš i kvůli \uv{pár adresám}.
Tuto situaci ovšem pomohlo vyřešit schéma CIDR (Classless Inter-Domain Routing). Toto schéma dovoluje libovolnou délku síťového přefixu (narozdíl od IPv4 class).
Velikost sítě je poté dána její maskou. Například pro síť o velikosti $2^18$ adres je její maska ve formátu 255.255.192.0 (CIDR formát - /18). Na posledních 18 bitech, které představují adresní prostor
jsou v masce 0 a na ostatních se nachází 1.

V rámci privátních sítí je možné použít tyto volně dostupné adresní prostory. Tyto privátní adresy neroutují do internetu, ale slouží pro routování v rámci NAT. Toto
bylo jedno z řešení problému s nedostatkem veřejných IP adres a také pomáhá izolovat privátní síť.

\begin{tabular}{c|c}
    10.0.0.0/8 & 16777216 adres ~ 1x class A \\
    172.16.0.0/12 & 1048576 adres ~ 16x class B \\
    182.168.0.0/16 & 65537 adres ~ 256x class C \\
\end{tabular}

\newpage

\subsection{Struktura hlavičky IPv4}

\begin{itemize}
    \item Version: 4 (IPv4)
    \item Header length: délka hlavičky v 32-bit slovech
    \item Type of Service: má různá využití, například pro QoS
    \item Total length: celková délka packetu v bytech
    \item Identification: identifikátor packetu (pro přiřazení fragmentů)
    \item Flags: příznaky, stavové flagy
    \item Fragment offset: relativní pozice fragmentu (jednotka: 8 bytů)
    \item Time to Live: prevence pro \uv{nekonečný packet}, jedná se o počet průchodů skrz routery, snižuje se 1, pokud klesne na 0, packet se zahodí
    \item Protocol: typ protokolu vyšší vrstvy (ssh, telnet, smtp)
    \item Header checksum: kontrolní součet hlavičky (slouží ke kontrole itegrity)
    \item Source address: zdrojová adresa
    \item Destination address: cílová adresa
    \item Options: nepovinné položky
    \item Padding: Zarovnání hlavičky na hranici 32 bitů
\end{itemize}

\subsection{Fragmentace v IPv4}

Paket nemůže být libovolně velký, je omezený délkou rámce (MTU - Maximum Transmission Unit), která je typicky 1500 bitů. Vyšší MTU redukuje overhead, nižší ovšem
snižuje transportní zpoždění. Pokud je ovšem paket větší, než je MTU, je nutné jej rozdělit na části. O toto se postará síťová vrstva a každý router může fragmentovat (rozdělovat)
IPvč pakety. Pakety jsou poté doručeny nezávisle na sobě a jsou složeny až na cílovém zařízení.

\section{Služby spojené se síťovou vrstvou}

\begin{itemize}
    \item ICMP (internet Control Message Protocol)
          Chybové zprávy sítě, které se odesílájí původnímu odesálateli. Odesílá protokol č. 1 v IP headeru, typ a kód (oba jsou 4 bity).
          Například: 0 - echo reply (ping), 3 - nedosažitelná adresa, 11 - TTL expired
    \item ARP - Address resolution Protocol (RFC826)
          Jedná se o protokol linkové vrstvy (2. úroveň ISO-OSI). Jedná se o mapování mezi síťovou a hardwarovou adresou (IP a MAC). To pomáhá zrychlovat komunikaci vytvořením
          rezoluční tabulky, každému záznamu je také přiřazena automaticky expirace záznamu.
\end{itemize}

\section{IPv6}

Důvody k přechodu do na nový adresový standard je několik. V roce 2011 byly rozdány poslední bloky IPv4 veřejných adres a již přidělené adresy nelze odebrat.
Nový standard pro IP tedy musí mít větší adresní prostor, také jsou tu určíté možnosti pro zautomatizování konfigurace (zbavení se DHCP) a lepší bezpečnosti.
Toto vše by mělo řešit IPv6.

\subsection{IPv4 vs IPv6}

IPv6 má o $2^96$ více adres než IPv4, stejně jako IPv4 používá hop limit, což je obdoba TTL. IPv6 má i rozdíli v hlavičce, nenachází se zde délka hlavičky, která je vždy
stejná. Také zde není fragmentace ani options. Chybí zde i kontrolní součet, který je vypočítáván na každém routeru. Oproti IPv4 je zde nová položka: Flow label (20b) - jedná se
identifikaci datového toku a slouží k usnadnění směrování, není zatím specifikováno. Hlavičku je možné rozšířit, za hlavičkou můžou následovat rozšiřující hlavičky, jejich pořadí je pevné.

V důsledku ztráty fragmentace na jednotlivých routerech, je nutné aby jí provedl již odesílatel. Pro IPv6 je minimální MTU 1280 bytů a pokud odesílatel odesílá větší balík, než je možné poslat
je mu odeslána zpráva pomocí ICMP.

\end{document}
