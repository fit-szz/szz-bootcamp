\documentclass{szzclass}

\usepackage{amsmath}
\usepackage{graphics}
% \usepackage{wrapfig}

% spacing
\usepackage{titlesec}
% \titlespacing*{\section}{0pt}{1ex}{0.5ex}
\titlespacing*{\subsection}{0pt}{1ex}{0ex}

\title{Metody řešení rekurentních rovnic, sestavování a   řešení rekurentních rovnic při analýze časové složitosti algoritmů.}
\renewcommand*\contentsname{Obsah}

\begin{document}
\maketitle

\tableofcontents
\newpage

\section{Rekurentní rovnice}

\subsection{Obecná rekurentní rovnice}

Lineární rekurentní rovnice řádu $k \in N$ je libovolná rovnice ve tvaru:

\begin{center}
$a_n+k + c_{k-1}(n) a_{n+k-1} + · · · + c_1(n) a_{n+1} + c_0(n) a_n = b_n$ pro každé $n \geq n_0$,
\end{center}


kde:
\begin{itemize}
    \item $n \geq n_0$
    \item $n_0 \in Z$
    \item $c_i(n) pro i = 0, . . . , k − 1$ jsou funkce $Z \rightarrow R$
    \item $c_0(n) \neq 0$
    \item $\{b_n\}^\infty_{n = n_0}$ (pravá strana rovnice)
    \item $\{a_n\}^\infty_{n=n_0}$ (řešení)
    \item pokud $\{\bar{a_n}\}^\infty_{n=n_0}$ je řešení, potom je $\{a_n\}^\infty_{n=n_0}$
    řešením této rovnice právě tehdy, když se dá zapsat jako
    $\{a_n\}^\infty_{n=n_0} = \{\bar{a_n}\}^\infty_{n=n_0} + \{\tilde{a_n}\}^\infty_{n=n_0}$,
    kde $\{\tilde{a_n}\}^\infty_{n=n_0}$ je nějaké řešení přidružené homogení rovnice.
\end{itemize}

\subsection{Rekurentní rovnice s konstantími koeficienty (LRRsKK)}

Lineární rekurentní rovnice řádu k s konstantními koeficienty je
libovolná rekurentní rovnice ve tvaru:

\begin{center}
$a_{n+k} + c_{k−1}a_{n+k−1} + · · · + c_1a_{n+1} + c_0a_n = b_n$ 
\end{center}

\begin{itemize}
    \item $n \geq n_0$
    \item $n_0 \in Z$
    \item $c_i \in R pro i = 0, . . . , k − 1$ jsou konstanty
    \item $c_0 \neq 0$
    \item $\{b_n\}^\infty_{n = n_0}$ (pravá strana rovnice)
    \item $p(\lambda) = \lambda^k + c_{k−1}\lambda^{k−1} + · · · + c_1\lambda + c_0$ je charakteristický polynom této rovnice
    \item $\lambda$ je chararistické, či vlastní číslo
    \item $\{\lambda\}^\infty_{n = n_0}$ je řešení homogení LRRsKK, pokud je $\lambda$ vlastní číslo této LRRsKK
    \item pokud existuje $k$ ruzných $\lambda_i$, potom $\{\lambda\}^\infty_{n = n_0}$ tvoří bázi prostoru řešení dané rovnice (stačí najít prvních $k$ členů)
\end{itemize}



\subsection{Moivre-ova věta}
$\alpha \pm i\beta = r[\text{cos}(\Phi) \pm i\text{sin}(\Phi)] \implies (\alpha \pm i\beta)^n = r^n[\text{cos}(n\Phi) \pm i\text{sin}(n\Phi)]$

\section{Řešení}

\subsection{Substituční metoda}

\begin{itemize}
    \item Odhadneme (uhádneme) tvar řešení (=indukční hypotéza).
    \item Pomocí matematické indukce nalezneme konstanty a ověříme
    správnosti odhadnutého řešení
\end{itemize}


\subsection{Iterační metoda}

\subsection{Mistrovská metoda}



\end{document}
