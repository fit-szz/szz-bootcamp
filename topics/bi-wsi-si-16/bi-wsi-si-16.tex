\documentclass{szzclass}
\usepackage{dependencies/szz-code}

\subject{PPA}
\code{BI-WSI-SI-16}
\topic{Logické programování, Prolog: fakta, pravidla, dotazy, způsob vyhodnocení dotazů, unifikace, operátor řezu.}

\begin{document}
\section{Logické programování}

\begin{itemize}
\item Založeno na predikátové logice
\item Vhodné zejména ve specifických aplikačních oblastech: databázové aplikace, automatické dokazování, umělá inteligence, \dots
\item Prolog
\item Deklarativní zápis programů, nezajímá se o složitost výpočtu
\end{itemize}

\section{Prolog}
Program Prologu se skládá z Hornových klauzulí. Skládají se z těla a hlavičky a mohou to být:
\begin{description}
\item[Fakta] - s hlavou bez těla, vždy pravdivý
\item[Pravidla] - s hlavou a tělem, pokud jsou splněny podmínky v těle, je pravdivý
\item[Dotazy] - bez těla
\end{description}

\begin{minted}{prolog}
hlavicka(jablko, jahoda). % fakt
hlavicka(Promenna,X) :- telo(X). % pravidlo
?- hlavicka(X,konstanta). % dotaz
\end{minted}

\begin{itemize}
\item Term
  \begin{itemize}
  \item Struktury - funktor a seznam argumentů \mintinline{prolog}{parent(josef,jan)}
  \item Jednoduché objekty
    \begin{itemize}
    \item Konstanty
      \begin{itemize}
      \item Atomy - začínají malým písmenem \mintinline{prolog}{jana}
      \item Čísla - \mintinline{prolog}{12 -34.5}
      \end{itemize}
    \item Proměnné - Začínají velkým písmenem \mintinline{prolog}{X, Name, _}
    \end{itemize}
  \end{itemize}
\end{itemize}

\subsection{Seznamy}
\begin{itemize}
\item \mintinline{prolog}{[] % prázdný seznam}
\item \mintinline{prolog}{[1]}
\item \mintinline{prolog}{[1,2,3]}
\item \mintinline{prolog}{[[1,2], 3]}
\item \mintinline{prolog}{[a | [b,c]] % oddělení hlavy od těla}
\end{itemize}

\subsection{Způsob vyhodnocení dotazů}
\begin{itemize}
\item Na program v Prologu lze nahlížet jako na databázi faktů a pravidel.
\item Vyhodnocuje se na principu procházení stavového prostoru, pravidla procházena DFS a tělo zleva doprava.
\item Řešení se hledá metodou backtrackingu.
\end{itemize}

\subsection{Unifikace}
Jsou tři různé operátory podobné rovnosti:
\begin{samepage}
\begin{itemize}
\item \textbf{\mintinline{prolog}{X is Y}} - Y se provede a unifikuje výsledek s X:
  \begin{itemize}
  \item \mintinline{prolog}{3 is 1+2} projde
  \item \mintinline{prolog}{1+2 is 3} selže.
  \end{itemize}
\item \textbf{\mintinline{prolog}{X = Y}} - unifikuje X a Y, bez provádění:
  \begin{itemize}
    \item \mintinline{prolog}{3 = 1+2} selže
    \item \mintinline{prolog}{1+2 = 3} selže.
  \end{itemize}
\item \textbf{\mintinline{prolog}{X =:= Y}} - obojí provede a porovná:
  \begin{itemize}
    \item \mintinline{prolog}{3 =:= 1+2} projde
    \item \mintinline{prolog}{1+2 =:= 3} projde.
  \end{itemize}
\end{itemize}
\end{samepage}

Dva termy se unifikují, když:
\begin{itemize}
\item X, Y jsou proměnné. X a Y se unifikuji a výsledek je kladný.
\item X, Y je-li jedna proměnná a druhá term. Substituuje term za proměnnou.
\item X, Y jsou stejné termy.
\item X, Y jsou struktury, tvořené stejným funktorem, počtem parametrů a jejich parametry si odpovídají.
\end{itemize}

\subsection{Operátor řezu}
\begin{itemize}
\item Řez fixuje přijaté „částečné řešení“ – omezuje splnění podcílů vlevo od řezu na jedinou možnost.
\item Překročení řezu zamezí využití ostatních pravidel.
\item Řez neovlivňuje zpětný chod vpravo do svého výskytu.
\end{itemize}

\begin{minted}{prolog}
% přidání prvku X na jeho začátek ovšem jen v tom případě, že X v L již není
% pridej(+X,+L,-NL) seznam NL vznikne ze seznamu L
pridej(X,L,L) :- prvek(X,L), % je-li X již prvkem L, nepřidám ho
  ! . % a zakáži návrat
pridej(X,L,[X|L]). % X není prvkem L (jinak bych se sem nedostal), mohu ho tedy přidat
\end{minted}

\subsection{Operátor fail}
\emph{"Jana má ráda muže, ale ne plešaté."}

Bez operátoru řezu to nejde. S ním a se standardním predikátem fail, který, je-li volán, okamžitě selže, ji sestavíme poměrně snadno:

\begin{minted}{prolog}
marada(jana,X) :- plesaty(X), % je-li X plešaté uspěje,
  !, % zakáže návrat
  fail. % a selže.
marada(jana,X) :- % k této klauzuli se výpočet dostane, pokud X není plešaté,
  muz(X). % je-li to muz, má ho Jana ráda
\end{minted}

\end{document}
