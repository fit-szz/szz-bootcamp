\documentclass{szzclass}

\title{Čisté objektové paradigma — klíčové pojmy, abstrakce, chování.}

\begin{document}
\maketitle

\tableofcontents
\newpage

\section{Základní pojmy}

\begin{itemize}
      \item \textbf{Třída} -- Vzor pro objekt. Definuje vlastnosti a chování objektů vytvářených podle tohoto vzoru.
      \item \textbf{Objekt} -- Instance třídy. Uchovává svůj vlastní vnitřní stav. Má vlastní identifikátor,
      kterým se odlišuje od ostatních objektů stejné třídy (objekt je vždy jedinečný v rámci programu).
      
      Chování objektu (\textit{behaviour}) je definované jeho třídou. Objekt reaguje na zasílané zprávy 
      (\textit{messages}) tak, že provede příslušnou metodu definovanou pro jeho třídu (\textbf{method lookup}).

      Objekt má zodpovědnost: sadu problémů, které řeší. (viz. \textbf{distribution of responsibility}, \textbf{separation of concerns}).

      \begin{itemize}
            \item \textbf{atributy} -- udržují vnitřní stav objektu
            \item \textbf{metody} -- využívají nebo mění vnitřní stav objektu
      \end{itemize}

      \item \textbf{Zpráva (Message)} -- \textit{What to do?}
      
      Zpráva odeslaná cílovému objektu, aby něco provedl. Různí příjemci mohou reagovat na stejnou zprávu různě.
      Příjemce je známý až při chodu programu (\textbf{late binding}).
      
      \item \textbf{Metoda (Method)} -- \textit{How to do it?}
      
      Sekvence instrukcí pro vyřešení problému. Definice, jak odpovědět na zprávu.
      Jakou metodu vybrat při danné zprávě se určuje dynamicky za běhu programu (\textbf{late binding}) podle procesu \textbf{method lookup}.
      
\end{itemize}

\section{OOP foundations}

Základy pro objektově orientované paradigma.

\subsection{Abstrakce}

Objekt vytváří určitou abstrakci, která schovává interní implementaci a detaily před ostatními objekty.
\textit{Ostatní objekty mohou s konkrétním objektem komunikovat pouze zasíláním zpráv a nesmějí vědět,
jak konkrétní objekt funguje uvnitř}.

\subsection{Encapsulation}

\textit{information hiding, zapouzdření}

Vnitřní stav objektu není přístupný zvenčí. Stav objektu je udržován pomocí jeho atributů.
Ty lze měnit a číst pouze pomocí metod. Tím má objekt větší kontrolu nad změnami svého stavu.

\subsection{Composition}

Třída může být závislá na jednom či více jiných objektech. To má za následek:
\begin{itemize}
      \item delegování problémů na jiné objekty pro lepší znovupoužitelnost kódu,
      \item výměna implementace dílčího objektu nijak neovlivní chování objektu, který s ním pracuje.
\end{itemize}

\subsection{Distribution of responsibility}

\textit{separation of concerns}

Rozdělení zodpovědnosti (funkcionality nebo chování) mezi objekty tak, aby se co nejméně překrývaly
(aby implementace zabrala co nejméně kódu, tedy znovupoužitelnost kódu byla co nejvyšší).

\subsection{Message passing}

\textit{delegating responsibility}


Delegování zodpovědoností na dílčí objekty.
Vyhodnocování metody objektu (\textit{receiver}) v kontextu jiného objektu (\textit{sender}).

\begin{itemize}
      \item \textbf{explicit} -- Předání \textit{sender} objektu do \textit{receiver} objektu.
      \item \textbf{implicit} -- \textit{Receiver} objekt provede \textit{method lookup} podle zaslané zprávy (Pharo).
\end{itemize}

\subsection{Inheritance}

Zakládání vlastností jedné třídy (\textit{child class}, \textit{derived class} nebo \textit{subclass}) na
jiné nadřazené třídě (\textit{parent class}, \textit{base class} nebo \textit{super class}).
Mezi třídami pak vzniká stromová hierarchie. \textit{Sub class} získává všechny vlastnosti svojí
\textit{super class} a přidává nové.

\paragraph{Polymorphism}
Jednotné rozhraní pro různé datové typy objektů. \textit{K objektům různých typů lze přistupovat pomocí jednoho
rozhraní}.
Dynamický polymorfismus: Jaký konkrétní objekt se použije se vybere až za běhu programu (\textbf{late binding}).


\section{OOP foundations ideas}

\paragraph{Uniform reference}
Všechno je objekt. Každá entita OO programu je objekt.

\paragraph{Uniform access}
Všechno se provádí pouze pomocí zasílání zpráv mezi objekty.


\end{document}
