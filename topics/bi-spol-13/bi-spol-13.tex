\documentclass{szzclass}
\usepackage{dependencies/szz-math}
\usepackage[czech]{babel}
\usepackage{bbm}

\newcommand*{\trans}[3]{{}^{#1}#2^{#3}}
\DeclareMathOperator{\sgn}{sgn}
\newcommand{\defect}[1]{\text{d} #1}
% \newcommand{\dim}[1]{\text{dim} #1}
% \newcommand{\ker}[1]{\text{ker} #1}

\subject{LIN}
\code{BI-SPOL-13}
\topic{Matice: součin matic, regulární matice, inverzní matice a její výpočet, vlastní čísla matice a jejich výpočet, diagonalizace matice.}

\begin{document}
\section{Součin matic}
Nechť $m,n,p\in \mathbb{N}$, $\mathbb{A}\in T^{m,n}$ a $\mathbb{B}\in T^{n,p}$. Součinem těchto matic je matice $\mathbb{D}=\mathbb{A}\mathbb{B}$, pro jejíž prvky platí:
$$
d_{ij}=\sum_{k=1}^{n}{a_{ik} b_{kj}}
$$

\section{Regulární a inverzní matice}
\begin{definition}[Regulární a inverzní matice]
Nechť $\mathbb{A}\in T^{n,n}$. Pokud existuje $\mathbb{A}\in T^{n,n}$ tak, že 
$$
\mathbb{A}\mathbb{B}=\mathbb{B}\mathbb{A}=\mathbb{E}
$$
Potom nazveme matici $\mathbb{A}$ \textbf{regulární} a matici $\mathbb{B}$ \textbf{inverzní}. Inverzní matici značíme $\mathbb{B}=\mathbb{A}^{-1}$.
\end{definition}

\begin{theorem}
Buď $\mathbb{A}\in T^{n,n}$. Následující tvrzení jsou ekvivalentní:
\begin{itemize}
\item $\mathbb{A}$ je regulární.
\item Soubor řádků matice $\mathbb{A}$ je LN.
\item $h(\mathbb{A}) = n$.
\item $\mathbb{A} \sim \mathbb{E}$.
\end{itemize}
\end{theorem}

\subsection{Výpočet inverzní matice}
Nechť $\mathbb{A} \in T^{n,n}$. Ověřte, zda je matice regulární a pokud je, nalezněte k ní matici inverzní $\mathbb{A}^{−1}$.

\begin{enumerate}
\item Hledáme matici $\mathbb{A}^{-1}$ s vlastností $\mathbb{A}^{-1}\mathbb{A} = \mathbb{A}\mathbb{A}^{-1} = \mathbb{E}$.
\item Doplněním zadané matice o jednotkovou matici stejného rozměru sestavme dvoublokovou rozšířenou matici $(\mathbb{A} | \mathbb{E}) \in T^{n,2n}$.
\item Na celou $(\mathbb{A} | \mathbb{E})$ používáme řádkové úpravy GEM, pro libovolnou posloupnost řádkových úprav realizovaných regulární maticí $\mathbb{P}$ pak platí
\[
  (\mathbb{A} | \mathbb{E}) \sim (\mathbb{P}\mathbb{A} | \mathbb{P}\mathbb{E}) = (\mathbb{P}\mathbb{A} | \mathbb{P}).
\]
Víme, že levou část je možné převést na jednotkovou matici právě tehdy, když je $\mathbb{A}$ regulární. Vznikne-li při úpravách $\mathbb{A}$ na horní stupňovitý tvar \textbf{nulový řádek}, pak $\mathbb{A}$ je singulární a \textbf{inverze neexistuje}.
\item Je-li $\mathbb{A}$ regulární, pak pro úpravy $\mathbb{P}$ vedoucí k převedení levého bloku matice $(\mathbb{A} | \mathbb{E})$ na jednotkovou matici platí $\mathbb{P} = \mathbb{A}^{-1}$, tedy $(\mathbb{A} | \mathbb{E}) \sim (\mathbb{E} | \mathbb{A}^{-1})$ a pravý blok výsledné matice obsahuje hledanou $\mathbb{A}^{-1}$.
\end{enumerate}

\section{Vlastní čísla}
\begin{definition}  
Řekneme, že $\lambda \in \mathbb{C}$ je \textbf{vlastní číslo operátoru} $A \in \mathcal{L}(V)$, právě když existuje $x \in V$ , $x \neq \theta$, takový, že $Ax = \lambda x$. Vektor $x$ pak nazýváme \textbf{vlastním vektorem operátoru} $A$ \textbf{příslušejícím vlastnímu číslu} $\lambda$. Množinu všech vlastních čísel $A$ nazýváme \textbf{spektrem} operátoru A a značíme symbolem $\sigma(A)$.

Analogicky pro matice $\mathbb{A}\in \mathbb{C}^{n,n}$, kde $\mathbb{A}=\trans{\varepsilon}{A}{}$.

Charakteristický polynom matice $\mathbb{A}$ (ozn. $p_\mathbb{A}$) definujeme předpisem $p_\mathbb{A}(\lambda) := \det(\mathbb{A} - \lambda E)$.
\end{definition}

\begin{definition}
Je-li $\lambda \in C$ vlastní číslo operátoru $A \in \mathcal{L}(V_n)$, pak podprostor $\ker{(A-
\lambda E)}$ nazýváme \textbf{vlastním podprostorem operátoru} $A$ příslušejícím vlastnímu
číslu $\lambda$.
\end{definition}

\begin{definition}
Nechť $\lambda \in C$ je vlastní číslo operátoru $A \in \mathcal{L}(V_n)$. Číslo $\defect{(A -
\lambda E)} = \dim{\ker{(A-\lambda E)}}$ nazýváme \textbf{geometrickou násobností} vlastního čísla $\lambda$ 
a značíme ji $\nu_g(\lambda)$.
\end{definition}

\begin{definition}
Nechť $A \in \mathcal{L}(V_n)$ a $\lambda \in \sigma(A)$. Násobnost čísla $\lambda$ jako kořene charakteristického polynomu $p_A$ operátoru $A$ nazýváme \textbf{algebraickou násobností} vlastního čísla $\lambda$ a značíme ji $\nu_a(\lambda )$.
\end{definition}

% \begin{definition}[permutace]
% Bijektivní zobrazení $\pi: \hat{n} \to \hat{n}$ nazveme \textbf{permutací} množiny $\hat{n}$.
% \begin{itemize}
% \item inverze permutace - dvojice $(\pi(i),\pi(j))$, kde $i<j$ a $\pi(i)>\pi(j)$, $i,j\in\hat{n}$.
% \item signum - číslo $\sgn\pi=(-1)^{I_\pi}$, kde $I_\pi$ je počet inverzí v permutaci $\pi$.
% \item množina všech permutací množiny $\hat{n}$ - $S_n$.
% \item transpozice permutace - permutace $\tau_{ij}\in S_n$, $\tau_{ij}(j)=i$, $\tau_{ij}(i)=j$ a pro ostatní $k$ $\tau_{ij}(k)=k$.
% \end{itemize}
% \end{definition}

\begin{definition}[determinant]
\textbf{Determinant} matice $\mathbb{A}\in \mathbb{C}^{n,n}$ je číslo:
\[
  \det{\mathbb{A}} = \sum_{\pi\in S_n} \sgn\pi \cdot a_{1\pi(1)} a_{2\pi(2)} \cdots a_{n\pi(n)}.
\]
\end{definition}

\subsection{Výpočet determinantu}
Determinant se dá vypočítat kombinací následujících postupů:
\begin{itemize}
\item Přes definici.
\item Sarrusovo nebo křížové pravidlo. (Sorosovo pravidlo zde asi nepomůže)
\item Je-li matice $\mathbb{A}$ trojúhelníková, lze determinant spočítat vynásobením čísel na diagonále.
\item $\det{\mathbb{A}^T}=\det{\mathbb{A}}^T$
\item Úprava GEM
  \begin{itemize}
  \item[$(G1)$] Prohození dvou řádků - mění znaménko determinantu.
  \item[$(G2)$] Vynásobení jednoho řádku nenulovým číslem - determinant se tím číslem musí vydělit.
  \item[$(G3)$] Přičtení k jednomu řádku $\alpha$násobek jiného řádku - determinant se nemění.
  \end{itemize}
\end{itemize}

\subsection{Výpočet vlastních čísel}
% Zadání \uv{nalezněte vlastní čísla a vlastní vektory operátoru/matice} znamená:
% \begin{itemize}
% \item nalézt spektrum $\sigma(A)$,
% \item ke každému vlastnímu číslu $\lambda$ nalézt bázi jeho vlastního podprostoru $\ker(A - \lambda E)$.
% \end{itemize}

\begin{itemize}
\item Pro danou matici $\mathbb{A} \in \mathbb{C}^{n,n}$ hledáme nenulové vektory $\mathbbm{x}$ a čísla $\lambda\in\mathbb{C}$ splňující rovnici
\[
  \mathbb{A}\mathbbm{x} = \lambda \mathbbm{x}.
\]
\item To je ekvivalentní hledání $\lambda$ takové, že homogenní soustava rovnic
\[
  (\mathbb{A} - \lambda E)\mathbbm{x} = \theta
\]
má nenulové řešení.
\item To nastává ale tehdy a jen tehdy (vzpomeňme Frobeniovu větu), když je matice $\mathbb{A} - \lambda \mathbb{E}$ singulární (neregulární).
\item A to je zase ekvivalentní tomu, že determinant matice $\mathbb{A} - \lambda \mathbb{E}$ je roven nule: abychom tedy našli vlastní číslo, řešíme rovnici
\[
  \det(\mathbb{A} - \lambda \mathbb{E}) = 0.
\]
\item Pro zadané vlastní číslo $\lambda$ už najdeme vlastní vektory snadno jako řešení homogenní soustavy uvedené výše.
\end{itemize}

\section{Diagonalizace matice}

\begin{definition}
Matice $\mathbb{A},\mathbb{B}\in \mathbb{C}^{n,n}$ nazveme \textbf{podobné}, právě když existuje regulární $\mathbb{P}\in \mathbb{C}^{n,n}$ tak, že:
\[
  \mathbb{A} = \mathbb{P}^{-1}\mathbb{B}\mathbb{P}
\]
\end{definition}

Ekvivalentně platí, že matice $\mathbb{A}, \mathbb{B} \in C^{n,n}$ jsou podobné právě tehdy, když jsou obě maticemi stejného lineárního operátoru (v nějakých bázích), tedy když existuje $\mathbb{A} \in \mathcal{L}(V)$ a báze $\mathcal{X}, \mathcal{Y}$ takové, že
\[
  \trans{\mathcal{X}}{\mathbb{A}}{} = \mathbb{A}
  \text{ a současně }
  \trans{\mathcal{Y}}{\mathbb{A}}{} = \mathbb{B}.
\]

Operátor $\mathbb{A} \in \mathcal{L}(V)$ nazveme \textbf{diagonalizovatelný}, jestliže existuje báze $\mathcal{X}$ prostoru $V_n$ taková, že matice $\trans{\mathcal{X}}{\mathbb{A}}{}$ je diagonální (matice je \textbf{diagonalizovatelná}, je-li podobná diagonální matici).

\begin{itemize}
\item Operátor $A \in \mathcal{L}(V)$ je diagonalizovatelný právě když $\forall\lambda_0 \in \sigma (A) : \nu_a(\lambda_0) = \nu_g(\lambda_0)$.
\item Libovolný soubor vlastních vektorů, ve kterém každý přísluší jinému vlastnímu číslu, je vždy LN.
\item Zadání \uv{ověřte, zda je operátor diagonalizovatelný, a nalezněte bázi, ve které je jeho matice diagonální} tedy znamená:
  \begin{itemize}
    \item nalézt spektrum $\sigma (A)$,
    \item ke každému vlastnímu číslu nalézt bázi vlastního podprostoru,
    \item porovnat algebraické a geometrické násobnosti u každého $\lambda \in \sigma (A)$,
    \item rovnají-li se pro každé $\lambda \in \sigma (A)$, bázi $\mathcal{X}$ sestavíme popořadě z bazických vektorů všech vlastních podprostorů. Matice přechodu $\trans{\mathcal{X}}{E}{\varepsilon}$ je bude obsahovat ve sloupcích, diagonální matice operátoru $\trans{\mathcal{X}}{A}{}$ bude na diagonále obsahovat v odpovídajícím pořadí všechna vlastní čísla (každé zopakované tolikrát, kolik je jeho násobnost).
  \end{itemize}
\end{itemize}

\section{Zdroje}
\begin{itemize}
\item Cvičení
\item Skripta
\end{itemize}

\end{document}
